\chapter{Multi-Functions}

Functions of the same name belong to an accordingly named set of functions. Such a function set is called a \textit{multi-function}. A function that belongs to a multi-function is also called an \textit{instance} of that multi-function. Multi-functions can be called like ordinary functions. The actual function being called is chosen at runtime as the function that is most specific in the set of functions fulfilling the subtyping relation comparing the actual (dynamic) argument types with each function's parameter types. We call this kind of invocation \textit{multiple dispatch}. 

Multi-Functions are useful, because they allow functions to be implemented with varying levels of generality. They lend themselves well to a varying ensemble of features and concerns:
\begin{itemize}
	\item \textbf{Single Dispatch} is supported natively, since multiple dispatch is a superset of single dispatch.
	\item \textbf{Intersection Types} composed of component, interface or semantic types can be used to define a sufficiently specific type for a given operation without referring to value types. Furthermore, varying degrees of increasingly specific intersection types can be used to specialise a function in different ways. This sufficiently specific type can be used to implement a function which would otherwise have been confined to a class hierarchy in object-oriented programming.
	\item \textbf{Dynamic specialisation and generalisation} of values can be used to specialise or generalise\footnote{Of course supposing that the generalised type still satisfies the type boundaries that its value is subject to.} the semantic type of a value at \textit{runtime}. Since the actual function being called is chosen at runtime when calling a multi-function, we can write functions that implement an operation for a given specialised type.
	\item \textbf{Extendability} is improved by the ability to define multi-functions \textit{across} files and compilation units. This supports features such as C$^\sharp$'s extension methods or Scala's implicit classes in a more concise way.
\end{itemize}

\noindent In this chapter, we will look at the the syntax of function declarations and define functions and multi-functions. After laying out the basics, we will define the rules of multiple dispatch and examine edge cases. We will see how intersection types, dynamic specialisation and generalisation, and extension methods can be used as proposed above. 



\section{Functions and Multi-Functions}

\begin{grammar}
\nt{func-def} &\produce \rh{\nt{func-head} [= \nt{expr}]\once} \\
\nt{func-head} &\produce \rh{\tn{function} \nt{id}\lpar\nt{func-params}\rpar [\tn{:} \nt{type}]\once} \\
\nt{func-params} &\produce \rh{\nt{func-param} [\tn{,} \nt{func-param}]\onceplus} \\
&\alt \rh{\nt{func-param}\once} \\
\nt{func-param} &\produce \rh{\nt{id} [\tn{:} \nt{type}]\once}
\end{grammar}

\begin{definition}
	A \textbf{function} $f$ is a mapping from an input type $\funcin(f)$ to an output type $\funcout(f)$. Each function has a full name, which we denote $\funcname(f)$. Note that the $\nt{id}$ of a function is not necessarily equal to its full name (we will introduce modules or packages in a future revision of the spec). The body of a function is an expression $\funcbody(f)$. The type of $\funcbody(f)$ must be a subtype of $\funcout(f)$. The body may be empty, in which case we write $\funcbody(f) = \none$ and call the function \textit{abstract}. An abstract function may not be called at run-time.

	The input type of a function $f$ is defined as follows: Let $[\tau_1, ..., \tau_n]$ be the list of parameter types for each parameter $p_i \in \{ 1, ..., n \}$. Then we have $\funcin(f) = (\tau_1, ..., \tau_n)$, i.e. an n-tuple over the list of parameter types.

	We denote the set of all possible functions as $\funcset$.
\end{definition}

\noindent Consider the following definition for a function add:
\begin{lstlisting}
    function add(a: Int, b: Int): Int = a + b
\end{lstlisting}
	
\noindent We will call the defined function $f$. We have the following properties:
\begin{align*}
	\funcname(f) &= \idval{add} \\
	\funcin(f) &= \type{(Int, Int)} \\
	\funcout(f) &= \type{Int} \\
	\funcbody(f) &= \expr{a + b}
\end{align*}

\begin{definition}
	A \textbf{multi-function} is a 2-tuple $\mathcal{F} = (n, F)$ where:
	\begin{itemize}
		\item $n$ is the name of the multi-function.
		\item $F$ is the set of functions belonging to the multi-function.
	\end{itemize}

	\noindent We define $F$ as follows: $F = \{ f \in \funcset \mid \funcname(f) = n \}$
\end{definition}

\noindent That is, a multi-function is a set of functions that share the same name. By convention, we sometimes write $f \in \mathcal{F}$ for $f \in F$. A function $f \in \mathcal{F}$ is also called an \textbf{instance} of $\mathcal{F}$. We denote the set of multi-functions as $\mfset$.

\bigskip

\noindent Consider the following function definitions:
\begin{lstlisting}
    function concat(x: ToString, y: ToString) = ... // f1
    function concat(x: List[a], y: List[a]) = ... // f2
    function concat(x: LinkedList[a], y: LinkedList[a]) = 
      ... // f3
\end{lstlisting}
	
\noindent Assuming no other function with the name \idval{concat} exists, we have the multi-function $\mathcal{F}_\idval{concat} = (\idval{concat}, F)$, where $F = \{ f_1, f_2, f_3 \}$ is the set of \idval{concat} functions defined above.



\section{Multiple Dispatch}

To define multiple dispatch formally, we first need an operation that allows us to reduce the set of multi-function instances to only those functions that could be called with a given tuple of arguments.

\todo{We should look at type parameters and include them in the dispatch. See: https://docs.julialang.org/en/release-0.4/manual/methods/#parametric-methods}

\todo{How do we treat functions with the same name but a different amount of parameters?}

\begin{definition}
	We define a function $\mffit : \typeset \rightarrow \mfset \rightarrow \powerset{\funcset}$ as follows:
	\begin{equation*}
		\mffit(t)(\mathcal{F}) = \{ f \in \mathcal{F} \mid \funcin(f) \geq t \}
	\end{equation*}
\end{definition}

\noindent That is, we look at all functions $f \in F$ and choose only those whose input type is a supertype of the given argument type $t$. We can not choose functions that have a more specific input type than the given argument type, because we need to call that function with valid arguments. We take functions with a more general input type into account, because such a function can be called with a subtype of the input type, i.e. with more specific arguments than needed.

\medskip

\noindent Suppose we have the multi-function $\mathcal{F}_\idval{concat}$. We get the following results when applying $\mffit$:
\begin{equation*}
	\mffit(\type{(String, String)})(\mathcal{F}_\idval{concat}) = \{ f_1 \}
\end{equation*}	
	
\noindent Only $f_1$ is chosen, because $\type{List[a]}$ is not a supertype of $\type{String}$ and neither is $\type{LinkedList[a]}$. $\type{ToString}$ is a supertype of $\type{String}$, since there is an implementation of $\mathtt{toString}$ for $\type{String}$. 
	
\medskip
	
\noindent Consider the next application of $\mffit$:
\begin{equation*}
	\mffit(\type{(LinkedList[Int], List[Int])})(\mathcal{F}_\idval{concat}) = \{ f_1, f_2 \}
\end{equation*}
	
\noindent Here, we choose $f_1$ and $f_2$, because the input types of both functions are supertypes of the argument types. $f_3$ is not chosen, since $\type{LinkedList[a]} \not\geq \type{List[a]}$ in the second argument type.
	
\medskip
	
\noindent Finally, let's consider the following application:
\begin{equation*}
	\mffit(\type{(LinkedList[Int], LinkedList[Int] \& Sorted)})(\mathcal{F}_\idval{concat}) = \{ f_1, f_2, f_3 \}
\end{equation*}
	
\noindent We choose all three functions, because $\type{ToString} > \type{List[a]} > \type{LinkedList[a]}$ and $ \type{LinkedList[a]} > \type{LinkedList[a] \& Sorted}$. The latter is true, because the intersection with $\type{Sorted}$ makes the $\type{LinkedList[a]}$ type more specific.


\begin{definition} \label{def:mfmin}
	Let $\mfmin : \powerset{\funcset} \rightarrow \powerset{\funcset}$ be the function defined as follows:
	\begin{align*}
		\mfmin(B) = \{ f \in B \mid \nexists f' \in B.\; \funcin(f') \leq \funcin(f) \}
	\end{align*}
\end{definition}

\noindent That is, $\mfmin$ is a function that extracts the most specific functions from a multi-function fit. Note that there may be multiple such functions, if their input types are not comparable, or none at all, if the fit is empty. We will explore both cases in the next example.

We will first look at an example with the multi-function $\mathcal{F}_\idval{concat}$. Suppose we have a fit $B = \{ f_1, f_2 \}$. We apply $\mfmin$ as follows:
\begin{equation*}
	Min(\{ f_1, f_2 \}) = \{ f_2 \} \text{, because } f_1 > f_2
\end{equation*}
	
\noindent A set with exactly one element is \textit{the} result that we need for multiple dispatch to be applicable. If the set was empty, we would not have found a suitable function to call. Perhaps even worse, if the set contains more than one element, we have an ambiguity and can not decide which function should be called. The following example shows that such an ambiguity exists.

\medskip
	
\noindent Assume we have the following two functions:
\begin{lstlisting}
    function area(x: Circle) = // f1
      pi * x.radius * x.radius 
    function area(x: +BoundingBox) = { // f2
      val b = x.BoundingBox
      val width = b.maxX - b.minX
      val height = b.maxY - b.minY
      width * height
    } 
\end{lstlisting}
	
\noindent That is, we can calculate the area both for a circle and for an object that has a BoundingBox component.\footnote{This is not a particularly nice example, since a BoundingBox should not be used to calculate an area, but let's just say that some incredibly wacky programmer decided to go with it.} We call the associated multi-function $\mathcal{F}_{\idval{area}}$. Now, what about a Circle that has a BoundingBox as a component? In other words, suppose we call the multi-function $\mathcal{F}_{\idval{area}}$ with an argument of the type $t = \type{Circle \& +BoundingBox}$. We have the following properties:
\begin{align*}
	B = Fit(t)(\mathcal{F}_{\idval{area}}) &= \{ f_1, f_2 \} \\
	Min(B) &= \{ f_1, f_2 \}
\end{align*}
	
\noindent The first result comes from the fact that for any types \type{a} and \type{b}, it holds that $\type{a} \geq \type{a \& b}$, and thus $\type{Circle} \geq \type{Circle \& +BoundingBox}$, as well as $\type{+BoundingBox} \geq \type{Circle \& +BoundingBox}$.
	
The second result is due to the fact that \type{Circle} and \type{+BoundingBox} are \textit{incomparable}, i.e. $\type{Circle} \nleq \type{+BoundingBox}$ and $\type{+BoundingBox} \nleq \type{Circle}$.
	
Since the most specific function is not unique, we should abort the compilation with an error (preferably) or even exit the program with a runtime error.
	
There are other instances of such an ambiguity, especially involving multiple function parameters (e.g. \type{(Int, a)} vs. \type{(a, Int)}), but the example above should be sufficient to illustrate the concept.
	
\medskip
	
\noindent In closing the example, we will look at the conditions needed to produce an empty fit. Let's assume we calculate the fit for a type $t = \type{Rectangle}$ in the multi-function $\mathcal{F}_{\idval{area}}$. Provided that Rectangle does not have a BoundingBox component, the fit will be empty, because neither \type{Circle} nor \type{+BoundingBox} are a supertype of \type{Rectangle}.

\bigskip

\noindent Having defined the $\mffit$ and $\mfmin$ functions, we can finally turn our attention to defining \textit{multi-function calls}.

\begin{definition}
	Suppose we have a function call expression as follows:
	\begin{equation*}
		N(e_1, ..., e_n)
	\end{equation*}
	
	\noindent A \textbf{multi-function call} is an operation with compile-time constraints and run-time semantics as defined in this section. Let $\mathcal{F}$ be the multi-function with the name $N$. Let $t = (t_1, ..., t_n)$ be the tuple type of the arguments with $e_1 : t_1, ..., e_n : t_n$. Let $B = \mffit(t)(\mathcal{F})$ be the fit of $\mathcal{F}$. Let $C = \mfmin(B)$ be the set of most specific functions in the fit $B$. 
	
	At \textit{compile-time}, we need to check whether the function to call would be unique if the run-time argument type was $t$. That is, if the run-time argument type would match the type bound $t$ exactly. We distinguish the following cases:
	\begin{itemize}
		\item If $C = \emptyset$, there is no function that fits the argument type. Thus, we throw an \textit{empty-fit} error.
		\item If $C = \{ f \}$, that is, there is exactly one most specific function, the multi-function call passes the compile-time check.
		\item If $C = \{ f_1, ..., f_k \}$ for $k \geq 2$, that is, we have an ambiguity, we throw an \textit{ambiguous-call} error.
	\end{itemize}
	
	\noindent At \textit{run-time}, the actual argument types might specialise the compile-time type bound $t$. Let $t' = (t'_1, ..., t'_n)$ be the run-time argument tuple type. Let $B' = \mffit(t')(\mathcal{F})$ and $C' = \mfmin(B')$. Since $C' \neq \emptyset$ at run-time\footnote{The \textit{empty-fit} error is always caught at compile-time. Refer to section \ref{section:empty-fit-ambiguous-call} for a proof.}, we only distinguish two cases:
	\begin{itemize}
		\item If $C = \{ f \}$, we call the function $f$. Note that $f$ cannot be abstract, because of the totality constraint on abstract functions, defined in section \ref{section:abstract-function-constraints}.
		\item If $C = \{ f_1, ..., f_k \}$ for $k \geq 2$, we throw an \textit{ambiguous-call} error.
	\end{itemize}
\end{definition}

\noindent Note that we have to distinguish between compile-time and run-time errors, which we will talk more about in the next section. Also note that the compile-time constraints do not refer to the abstractness of the function $f$, since the point of an abstract function is exactly that the compile-time checks pass while we require specialisation of the argument types at runtime.



\section{Empty-Fit and Ambiguous-Call Errors} \label{section:empty-fit-ambiguous-call}

In the definition of \textit{multi-function calls} we had to distinguish between compile-time and run-time errors. In particular, we defined the \textit{empty-fit} error as compile-time-only, while the \textit{ambiguous-call} error may either occur at compile-time or run-time. Of course, compile-time errors are always preferable to run-time errors. However, with the expressiveness of the type system of Lore, we can not get around the possibility of a run-time error for the ambiguity case. We will see why in this section.

\bigskip

\noindent First, we will look at the compile-time-only property of the \textit{empty-fit} error. We will prove that the error can only occur at compile-time by looking at a call of a multi-function $\mathcal{F}$. Let $t$ be the tuple type of the argument values deduced at compile-time. Let $B = \mffit(t)(\mathcal{F})$ and $C = \mfmin(B)$. Let's assume we call the multi-function at run-time with an argument type $t'$. Let $B' = \mffit(t')(\mathcal{F})$ and $C' = \mfmin(B')$. 

Assume that $C' = \emptyset$, but $|C| = 1$.\footnote{We can safely assume that $|C| \leq 2$, since otherwise the code would not compile because of an ambiguity error.} That is, calling the multi-function at compile-time was valid, but we can't find any fitting function to call at run-time. Since $C \neq \emptyset$, we have $B \neq \emptyset$, so there exists an $f \in B$ such that $\funcin(f) \geq t$ (by definition of the fit). Now, we observe that $t'$ specialises the argument type $t$, so that $t \geq t'$ holds. This is the only way in which the run-time type of the argument can change. In particular, $t$ can not be generalised, because it is the upper-bound for any actual argument types and thus generalising $t$ would lead to a type-checking error. From $t \geq t'$, it follows, by transitivity of the subtyping relation, that $\funcin(f) \geq t'$ and thus $f \in B'$ (by definition of the fit). Finally, we can derive that $f \in C'$, which contradicts $C' = \emptyset$. Thus we have proven that the \textit{empty-fit} error can only occur at compile-time.

\bigskip

Moving on to the \textit{ambiguous-call} error, we can show with a simple example that such an error might only be caught at run-time. Let's look at the ambiguity example below definition \ref{def:mfmin} again. We have two functions $f_1 : \type{Circle \fto Real}$ and $f_2 : \type{+BoundingBox \fto Real}$. At compile-time, in an expression $\idval{area}(c)$, $f_2$ is the unique most specific function to call given a variable $c : \type{+BoundingBox}$. Since $\type{Circle \& +BoundingBox}$ is a subtype of $\type{+BoundingBox}$, we may safely pass such an argument at run-time. The problem is that the additional intersection type allows both \texttt{area}-functions to be called, as can be seen in the original example, instead of only the one that takes a \type{+BoundingBox} argument. This leads to an ambiguity error at \textit{run-time}.

More generally, we can observe that intersection types are precisely the feature that make our version of multiple dispatch ambiguous.\todo{Proof needed.} If we have two functions $f_1 : \type{t1}$ and $f_2 : \type{t2}$ that have the same name, we can introduce a run-time ambiguity with the intersection type \type{t1 \& t2}.



\section{Constraints on Return Types}

So far we have only talked about \textit{parameter types} of multi-function instances. We also have to account for return types of functions, because we must ensure type safety.

First of all, we can not incorporate the return type into the multiple dispatch process. Doing so would jeopardise our type inference mechanism, which will frequently rely on function return types. To ensure type safety, we will impose a constraint on return types.

\begin{definition} \label{def:constraint-return-types}
	Let $\mathcal{F}$ be a multi-function. The following \textbf{constraint on return types} must be satisfied for all $f, f' \in \mathcal{F}$:
	\begin{equation*}
		\funcin(f) \geq \funcin(f') \implies \funcout(f) \geq \funcout(f')
	\end{equation*}
\end{definition}

\noindent That is, assume $f'$ specialises $f$. If we have a value $v$ with a compile-time type of $\funcin(f)$ in a function call expression $f(v)$, we can assume a return type of $\funcout(f)$. Since $f'$ is called instead of $f$ if $v$ is run-time specialised to $\funcin(f')$, we must ensure that the return type of $f'$ satisfies the type bound $\funcout(f)$. Otherwise, a value might be returned that does not fit the type that was deemed valid at compile-time.

If a multi-function fails to satisfy this constraint, an \textit{invalid-return-type} error will be thrown at compile-time, mentioning all offending function definitions.



\section{Abstract Function Constraints and Usage} \label{section:abstract-function-constraints}

We want abstract functions to provide the guarantee that multiple dispatch always finds a concrete function to call at run-time. To achieve this, we must ensure the following two properties:
\begin{enumerate}
	\item Abstract functions must be fully implemented, which is formalised in the \textbf{totality constraint}.
	\item The input type of an abstract function must be abstract itself, which is formalised in the \textbf{input abstractness constraint}.
\end{enumerate}

\begin{definition} \label{def:totality-constraint}
	Let $\mathcal{F}$ be a multi-function. The following \textbf{totality constraint} must be satisfied for all abstract functions $f \in \mathcal{F}$:
	\begin{align*}
		\forall s < \funcin(f).\; \neg \typeabstract(s) \implies \exists f' \in \mathcal{F}.\; \neg \funcabstract(f') \land f' \in \mffit(s)(\mathcal{F})
	\end{align*}
\end{definition}

\noindent That is, all concrete subtypes $s$ of $f$'s input type must be covered by at least one concrete function. This ensures that all input values with which the function can ever be called are dispatched to a concrete function.

If a multi-function $\mathcal{F}$ does not satisfy the totality constraint, a \textit{missing-implementation} error is thrown, which includes a list of input types that need to be covered.

\begin{definition} \label{def:input-abstractness-constraint}
	Let $\mathcal{F}$ be a multi-function. The following \textbf{input abstractness constraint} must be satisfied for all abstract functions $f \in \mathcal{F}$: 
	\begin{align*}
		\funcabstract(\funcin(f))
	\end{align*}
\end{definition}

\noindent To appreciate the need for this constraint, consider the following: Assume we define an abstract function $f \in \mathcal{F}$ with an input type \type{A}, which is a concrete record type. The type \type{A} has two direct subtypes \type{A1} and \type{A2}. Even if we satisfied the totality constraint for $f$ by defining two implementations $f_1$ and $f_2$, we could call $\mathcal{F}$ with an argument $a : \type{A}$, which would attempt to call $f$, because $f_1$ and $f_2$ are not contained in the fit of the call. Thus, we would attempt to call an abstract function at run-time, contradicting the goal stated above.

If an abstract function $f$ does not satisfy the input abstractness constraint, an \textit{input-type-not-abstract} error is thrown.

\medskip

\noindent We will finish this section with a usage example. Suppose we have the following types:
\begin{lstlisting}
    type T = A | B
    record A { ... }
    record B { ... }
\end{lstlisting}

\noindent That is, \texttt{T} is a sum type, while \texttt{A} and \texttt{B} are concrete record types. We also have the following two functions:
\begin{lstlisting}
    function f(a: A): A = ...
    function f(b: B): B = ...
\end{lstlisting}

\noindent Suppose we have a value \texttt{v} of type \texttt{T}. If we try to call \texttt{f(v)}, we will get an empty fit error at compile-time, because both functions are too specific for the general type \texttt{T}, even though at run-time, we are \textbf{certain} that one of the two functions must be called, because \texttt{v} must either have the type \texttt{A} or \texttt{B}.

This is one instance where abstract functions are useful. We define the following additional function to fix the problem:
\begin{lstlisting}
    function f(t: T): T
\end{lstlisting}

\noindent This function is obviously abstract, because it has no body. It satisfies the \textbf{totality constraint}, since the concrete subtypes \texttt{A} and \texttt{B} each have an associated concrete function. The \textbf{input abstractness constraint} is also satisfied, because sum types are by definition abstract.\footnote{This may not be so obvious, but there is no way to instantiate a value of a sum type directly: You always have to instantiate the value corresponding to one of the subtypes.} We can now call the function \texttt{f} with a value \texttt{v} of type \texttt{T} without getting a compilation error. The call should lead to the expected semantics. 

Note that the \textbf{constraint on return types} from definition \ref{def:constraint-return-types} is also satisfied:
\begin{enumerate}
	\item For \texttt{f(a: A)}: $\type{T} \geq \type{A}$.
	\item For \texttt{f(b: B)}: $\type{T} \geq \type{B}$.
\end{enumerate}

\noindent This concludes our usage example about abstract functions.



\section{Totality Constraint Verification}

It is not obvious how to check the totality constraint defined in the last section. In this section, we will develop an algorithm that can verify the constraint for an arbitrary abstract function.

Following the definition of the totality constraint, the crux of the algorithm is to identify the set of concrete subtypes that need to be checked. The check itself is trivial, since it only involves finding a concrete function that fits the subtype.

We will see that this is entirely possible. First, we will define a lemma that will allow us to ignore all added intersection types.

\begin{lemma} \label{lemma:totality-constraint-intersection-fit}
	Let $\mathcal{F}$ be a multi-function and $f \in \mathcal{F}$ an abstract function. Let $s < \funcin(f)$ be a concrete subtype for which a fitting concrete function $f'$ exists. Then for any type $t$, the subtype $s \mathintersect t < \funcin(f)$ does also have $f'$ as a fitting function.
\end{lemma}

\begin{proof}
	Trivial.\footnote{No, actually, this is a TODO. ;)} $\blacksquare$
\end{proof}

\noindent This lemma allows us to ignore incredibly large sets of intersection types when building the set of subtypes to be checked.

\begin{definition} \label{def:totality-constraint-verification-algorithm}
	We define the function $\verifytotality$ that verifies a multi-function's totality constraint with the following algorithm:
	
	\begin{lstlisting}
    function verify-totality(F: multi-function):
      F filter f => abstract(f)
        forall f => is-verified(F, f)
      
    function is-verified(F: multi-function, f: function):
      find-direct-declared-subtypes(in(f))
        forall s => 
          F exists f' => 
                in(f') < in(f)
            and (not abstract(f') 
                  or abstract(f') and is-verified(F, f))
            and fit(s)(F) contains f'
    
    function find-direct-declared-subtypes(tt: tuple-type):
      tt as sequence
        map t => if abstract(t) 
                 then direct-declared-subtypes(t) 
                 else [t]
        product
	\end{lstlisting}
\end{definition}

\noindent In the algorithm above, direct declared subtypes are types that are explicitly marked as direct subtypes, such as immediate subclasses. In the function \texttt{find-direct-declared-subtypes}, a product of all individual direct declared subtypes is built to get tuple types that subtype the original input type.

The central idea for proving (and improving) this algorithm is the observation that only looking at the directly declared subtypes is sufficient to prove the totality constraint for all subtypes. One step in this direction is lemma \ref{lemma:totality-constraint-intersection-fit}.

Requiring \texttt{in(f') < in(f)} in \texttt{is-verified} uses the strict hierarchy of function input types to ensure that there is no infinite recursion.



\section{Multi-Functions and Intersection Types}

We can use intersection types to refine a value's type and dispatch based on the presence of such an intersection type at run-time. For example, we could define a \texttt{search} function on a list as follows:
\begin{lstlisting}
    function search(list: List[a]): a =
      // search with linear search
    function search(list: List[a] & Sorted): a =
      // search with binary search
\end{lstlisting}

\noindent That is, in the type system we can encode that some lists are sorted, and for those lists we can specialise the \texttt{search} function so that it uses a more efficient algorithm, that is, binary search instead of linear search.

\medskip

\noindent Another use case for intersection types in the context of multiple dispatch arises from the concept of components. Suppose we have a function \texttt{density} that calculates the population density of an object, which needs to have the components \texttt{+Population} and \texttt{+Area}. That property can be modelled with an intersection type \texttt{+Population \& +Area}. The function is thus defined as follows:
\begin{lstlisting}
    function density(v: +Population & +Area): Float = 
      v.Population / v.Area
\end{lstlisting}

\noindent No matter whether the object is a city, a country or a planet, the density function will be able to calculate the density as long as the object has a population and an area.

Suppose we have objects where the total amount of water area is known. We can specialise the function \texttt{density} to include this circumstance:
\begin{lstlisting}
    function density(v: +Population & +Area & 
        +WaterArea): Float =
      v.Population / (v.Area - v.WaterArea)
\end{lstlisting}

\noindent In cases where \texttt{+WaterArea} is present, the dispatcher chooses the second function, which makes the calculation more accurate. If \texttt{+WaterArea} is not present, the original \texttt{density} function is called instead.



\section{Dynamic Specialisation and Generalisation}

\todo{Intersections with semantic types can be used to define a state, as used in Papyrus script (https://www.creationkit.com/index.php?title=States_(Papyrus)).}

Each value in Lore is tagged with a core value type and additional semantic types. Dynamic specialisation and generalisation are features that allow us to \textit{safely} add and remove semantic types to/from a value at runtime.

Let's look at the lists example again. We may have the following \texttt{sort} function:
\begin{lstlisting}
    function sort(list: List[a]): List[a] & Sorted =
      val list2: List[a] = ... // Sort the list.
      list2.attach[Sorted]
\end{lstlisting}

\noindent This function sorts the list and subsequently \textit{attaches} the semantic type \texttt{Sorted} to it. A subsequent call to \texttt{search} (as defined above) will then multiple-dispatch to the more efficient method.

(One problem with this approach is that we have to manually remove the \texttt{Sorted} type if the list becomes somehow unsorted. This is really hard on the programmer, because he or she has to track this additional state at least while writing the lists API. Maybe we can automate that or at least allow specifying invariants that are checked at the beginning and end of a method?) \todo{Manual removal of Sorted type.}

It may not always be so simple. Suppose we have a type $\type{Rectangle} \mathsum (\type{Circle} \mathintersect \type{Red})$ with $\type{Red}$ being a semantic type. We want to remove the semantic type. For that, we can use pattern matching on types:
\begin{lstlisting}
    val shape: Rectangle | (Circle & Red)
    val shape2 = shape.as[a | (b & Red) => a | b]
    // shape2: Rectangle | Circle
\end{lstlisting}

\noindent We can also remove all occurrences of any semantic type's subtypes from a complex type:
\begin{lstlisting}
    val shape: (Rectangle & Blue) | (Circle & Red)
    val shape2 = shape.remove[Color]
    // shape2: Rectangle | Circle
\end{lstlisting}



\section{Implementing Extension Methods}

\href{https://en.wikipedia.org/wiki/Extension_method}{Extension methods} can be implemented in Lore without requiring a special syntax, because functions aren't tied to types, but still support single and multiple dispatch. For example, we could extend the \type{String} type with a \texttt{reverse} function:
\begin{lstlisting}
    function reverse(str: String): String = ...
\end{lstlisting}

\noindent The "extension method" can be defined anywhere. It can be used like a normal function in any part of the code, assuming it's imported to a given scope.








